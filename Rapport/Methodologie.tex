\section{Model methodologie}\label{Sect_Methodologie}
\subsection{Incurred but not reported}
	The task of the actuarial model is to predict the IBNR, the incurred but not reported claims. The IBNR can be divided into 3 distinct elements, which we defined as pure IBNR, IBNER and unpure IBNR. Pure IBNR are claims which are not reported at the observation date, meaning the insurer has no information on them. The insurer only knows that a claim happened. IBNER, incurred but not enough reported, are claims which have been reported and the insurer the information on the claims in their database. Unpure IBNR consist of claims which might reopen at any given time. This mean that a claim which closed in 2017 might reopen in 2018 or 2019. Unpure IBNR is a small proportion of the total IBNR, but still should be considered in the model.
	The actuarial department uses a modifier chain-ladder method for their model. We will try a more hierarchical approach, where we cluster our data in more homogeneous groups. First, we develop a model for each of the three IBNR types. Our team focuses on the IBNER part, while the pure and unpure IBNR models are still chain-ladder based and were developed by the actuarial department. For the IBNER model, we grouped the data according the following claims characteristics:  total loss, total loss without replacement cost endorsement, luxury repairable vehicles, non luxury non rental repairable vehicles and non luxury rental repairable vehicles. We suppose that the frequency and severity distributions are very similar within these groups. How each group is treated, will be discussed later. 
