\section{Analyse de la fréquence}\label{Section_Frequence}
	Lors de l'analyse du capital à investir pour une compagnie de réassurance, il va de soit que le dénombrement des sinistres à une importance toute aussi grande que pour la modélisation de la sévérité selon la \textit{théorie des modèles collectifs du risque}. \\
	
	Afin d'analyser cette composante, \cite{Parodi2015_Pricing_in_GenIns}(p.187-206) suggère une approche selon les lois de dénombrement classiques telles que la loi binomiale, binomiale négative et la loi de Poisson. Cependant, cette approche ne tient pas compte des tendances qu'il peut y avoir à travers le temps.\\
	
	Ainsi, afin de pouvoir étudier l'évolution du dénombrement des sinistres dans le temps, il faut utiliser les processus stochastiques. Dans cette perspective, \cite{Mikosch_PoissProcess2009}(p.7-9) recommande d'utiliser les processus de Poisson.\\
	
	Dans cette section, les processus de Poisson homogènes et non homogènes sont donc testés afin de modéliser le dénombrement des sinistres d'une année à l'autre.
	
	\begin{Definition}
		Soit $\{ N(t)\in \mathbb{N};t>0 \} $, un processus de comptage dont les accroissements sur l'intervalle de temps $(s,t]$ sont définis par $\{ N(t)-N(s),\:t>s\}$. Si on définit que les temps inter-sinistres suivent une loi exponentielle, ce processus de comptage obéit à une loi de Poisson dont la fonction de masse de probabilités est définie par
		\begin{align}
		P\left[N\left(t\right)-N\left(s \right) = x \right] = \frac{\left[\Lambda(t) - \Lambda(s)\right]^x e^{-\left[\Lambda(t) - \Lambda(s)\right]}}{x!}  \label{ProcessusPoisson}
		\end{align}
		et l'espérance des accroissements, $\Lambda \left( t\right) -\Lambda \left( s\right) $, correspond à 
		\begin{align}
		E[N(t)-N(s)] = \Lambda(t) - \Lambda(s) = \int_{s}^{t} \lambda(u)\textrm{d}u \;  ,\ 0<s<t, \label{Esp_Accroissements}
		\end{align}
		où $\lambda$ correspond au taux d'intensité du processus. 
	\end{Definition}

	Dans le présent travail, afin d'estimer les paramètres des processus de Poisson, la méthode du maximum de vraisemblance est utilisée puisque celle-ci comporte des propriétés théoriques favorables (voir \cite{Edward_MaximumLikelyhood1988}, p.347-362). \\
	
	\subsection{Processus de Poisson homogène}
		 Pour débuter, le processus de Poisson homogène possède un taux d'intensité constant. Ainsi, afin d'en estimer son paramètre, il suffit de maximiser la fonction de vraisemblance tel qu'elle est définit par
		\begin{align}
			\mathcal{L}\left(\lambda\right)  
			&= \prod_{i=1}^{n} f_{N(t)_i} \left( x_i, \lambda, t \right) \nonumber\\
			&= \prod_{i=1}^{n} \frac{ \left( \lambda t \right)^{x_i} e^{-\lambda t} }{x_i!}, \label{Fct_Vraiss_PoissHomo}
		\end{align}
		où $n$ correspond au nombre d'observations de la base de données.\\
		Cependant, il est plus aisé de maximiser la log-vraisemblance qui est
		\begin{align}
			l \left(\lambda \right)  
			&= \sum_{i=1}^{n}  \left( x_i \ln\left(\lambda t\right)-\lambda t - \ln \left(x_i!\right)  \right) \nonumber\\
			&= n \bar{x} \ln \left( \lambda t \right) - n\lambda t - \sum_{i=1}^{n}\ln \left(x_i!\right). \label{Fct_logVraiss_PoissHomo}
		\end{align}
		En dérivant \ref{Fct_logVraiss_PoissHomo} en fonction de $\lambda$ et en égalisant le résultat à zéro, on obtient $\hat{\lambda}$, l'expression de l'estimateur du maximum de vraisemblance de $\lambda$, soit
		$$	\hat{\lambda} = \bar{x}/t. $$
		
		Ainsi, selon la méthode du maximum de vraisemblance, le meilleur estimateur du paramètre $\lambda$ d'une loi de Poisson homogène est la moyenne du nombre de sinistre dans l'intervalle de temps $t$, divisé par ce dernier.
		
	\subsection{Processus de Poisson non homogène}
		Lorsque l'intensité d'un processus de Poisson n'est pas constant dans le temps, il survient que les accroissements ne sont pas stationnaires. Cela signifie que si ce processus débute au temps 0, il n'aura pas la même intensité que s'il débute au temps $t$, pour $t>0$. On obtient alors des processus légèrement plus complexes à modéliser. \\
		
		Parmi l'infinité de modèles qui existent pour capturer ces tendances au fil du temps, quelques-un sont proposés par \cite{Kuhl_PoissonNonHomo_Trends_1997}. Cependant, parmi ceux-ci, certains donnent des résultats douteux sur le long terme. Ce qui amène à considérer qu'il faut faire attention à ne pas faire du surapprentissage statistique. C'est à dire que le modèle n'est bon que pour modéliser les données qui ont servi à l'entraîner. Pour cette raison, les modèles trop complexes perdent du pouvoir prédictif.\\
		
		Pour ce travail, les modèles ont été sélectionnés parmi ceux proposées dans \cite{TheorieDuRisque2018_MarceauEtCossette}(p.231). Ainsi, la fonction linéaire, la fonction de puissance et la fonction périodique avec effet saisonnier sont les modèles d'intensité retenus.
		
		\subsubsection{Intensité linéaire}
			Le processus de Poisson avec intensité linéaire est celui qui permet de modéliser une augmentation de l'intensité qui est constant dans le temps comme dans le cas des bases de données \texttt{norwegianfire} (graphique \ref{Graph_Norwegianfire_FreqCumul}) et \texttt{danish} (graphique \ref{Graph_Danish_FreqCumul}).
			\begin{Proposition}
			 	Soit $\lambda(u) = a+bu \ , \; a>0,\,b\geq 0, u>0$. Alors, on obtient 
				\begin{align}
				\Lambda(t+1) - \Lambda(t) = a+\frac{b(2t+1)}{2}. \label{Intensite_lineaire}
				\end{align}
			\end{Proposition}
			\textit{Preuve:} L'expression en \ref{Intensite_lineaire} est obtenue directement en appliquant \ref{Esp_Accroissements}.\\
			
	 		Il s'ensuit que la fonction de vraisemblance est définie par 
		 	 	\begin{align}
		 	 		\mathcal{L}\left( a,b \right) = \prod_{i=1}^{n} \frac{ \left( a+\frac{b(2t+1)}{2} \right)^{x_i} e^{-(a+\frac{b(2t+1)}{2})} }{x_i}. \label{Vrais_Poiss_Homo}
		 	 	\end{align}
				
			Puis la fonction de log-vraisemblance est
			\begin{align}	
				l\left(a,b\right) 
				&= \sum_{i=1}^{n} \left( x_i \ln \left( a+\frac{b(2t+1)}{2} \right) -a-\frac{b(2t+1)}{2} -\ln x_i   \right) \nonumber \\
				&=n\bar{x}\ln \left( a+\frac{b(2t+1)}{2} \right) - na - \frac{nb(2t+1)}{2} -\sum_{i=1}^{n}\ln x_i. \label{Log-Vrais_Poiss_Homo}
			\end{align}
			
			À ce stade, le système d'équations qui découle de \ref{Log-Vrais_Poiss_Homo} peut être résolu numériquement.\\
			
			Dans son mémoire de maîtrise, \cite{Drazek_PoissProcess2013}(p.32-33) a développé un algorithme pour trouver les paramètres de processus de Poisson. Dans son ouvrage, il minimise la log-vraisemblance négative puisque la majorité des commandes d'optimisation numériques sont programmées pour minimiser des fonctions. \\
			
			Le code \texttt{R} qui a été utilisé pour estimer les paramètres des processus de Poisson non homogène avec intensité linéaire est présenté dans le code informatique \ref{Code_ProcessusPoisson_homo}.
			\begin{Code} \label{Code_ProcessusPoisson_homo}
				\begin{verbatim} 
				
				neg_log_vrais <- function(para){
				-sum( log( dpois(data, para[1] + para[2] * (2*t+1)/2)))
				}
				
				mle <- constrOptim(c(0.05, 0), neg_log_vrais, grad = NULL, 
				ui = c(1,0), ci = 0,outer.eps = .Machine$double.eps)
				\end{verbatim}
			\end{Code}
			Afin de trouver des valeurs de départ pour cet outil d'optimisation, une façon appropriée est d'utiliser la méthode des moments qui est proposé dans \cite{LossModels_Klugman2012}(p.253-255).\\
			
			À noter que les fonctions \texttt{optim} et \texttt{constrOptim} donnent des résultats plus précis lorsque les valeurs de départs sont proches du résultat attendu.
			En changeant ces paramètres initiaux, les deux fonctions retourneront des résultats qui peuvent beaucoup fluctuer.
					
		\subsubsection{L'intensité est une fonction de puissance}
			Advenant le cas où l'augmentation de l'intensité se fait de façon plus prononcée qu'avec un modèle linéaire, il est possible d'utiliser une fonction de puissance. Dans le cas présent, cette fonction s'inspire de la loi de Weibull.
			
			\begin{Proposition}
				Soit $\lambda(u) = (\beta u)^\tau \ , \; \beta>0,\,\tau>0, u>0.$ Alors on a 
				\begin{align}
				\Lambda(t+1) - \Lambda(t) = \frac{\beta^\tau}{\tau+1}((t+1)^{\tau+1}-t^{\tau+1}) \label{Intens_Poiss_Puissance}.
				\end{align}
			\end{Proposition}
			\textit{Preuve:} L'expression en \ref{Intens_Poiss_Puissance} est obtenue directement en appliquant \ref{Esp_Accroissements}.\\
			
			Puis, la fonction de vraisemblance est définie par
			\begin{align}
			\mathcal{L}\left( x,\beta,\tau,t \right) 
			= \prod_{i=1}^{n} \frac{ \left( \frac{\beta^\tau}{\tau+1}((t+1)^{\tau+1}-t^{\tau+1}) \right)^{x_i} e^{-(\frac{\beta^\tau}{\tau+1}((t+1)^{\tau+1}-t^{\tau+1}))} }{x_i}.\label{Vrais_Poiss_Puissance}
			\end{align}
			
			Dans ce cas-ci, pour trouver les estimateurs des paramètres, le plus simple est d'utiliser la méthode algorithmique de la même façon que pour le cas précédent.\\
			
			Par ailleurs, afin de simplifier l'algorithme, il est possible de réécrire l'intensité cumulée sous la forme
			
			\begin{align}
				\Lambda(t+1) - \Lambda(t) 
				&= \frac{\beta^\tau}{\tau+1}((t+1)^{\tau+1}-t^{\tau+1}) \nonumber\\
				&= \beta^* ((t+1)^{\tau^*} - t^{\tau^*}), t>0, \label{Intensite_Fonctio_Puissance}
			\end{align} 
			
			où $\beta^*=\beta^\tau$ et $\tau^*=\tau+1$.\\
			
			L'algorithme \texttt{R} permettant de procéder à cette optimisation est reproduit dans le code informatique \ref{Code_Parametres_PoissPuissance}.
			
			\begin{Code}\label{Code_Parametres_PoissPuissance}
			\begin{verbatim}
			
				intensite <- function(t,para){
				(para[1] * t)^para[2]
				}
				
				neg_log_vrais <- function(para){
				-sum( log( dpois(data, intensite(t+1,para) - intensite(t, para))))
				}
				
				mle_nonhomo_wei <- constrOptim(c(1, 1), neg_log_vrais, grad = NULL, 
				ui = c(1,0), ci = 0,outer.eps = .Machine$double.eps)
			\end{verbatim}
		\end{Code}
			
		\subsubsection{Intensité périodique}
			Finalement, le modèle d'intensité périodique permet de capturer une tendance cyclique comme dans le cas de \texttt{secura} avec le graphique \ref{Graph_Secura_FreqCumul}.
			\begin{Proposition}\label{Intens_Periodique}
			Soit $\lambda(u) = a+b \cos\left(\frac{2\pi u}{c}\right) \ , \; a>0,\,b \in \left[0,a\right],\,c>0, u>0. $ Alors, on a\\
			\begin{align*} 
				\Lambda(t+1) - \Lambda(t) = a-\frac{bc}{2\pi}\left[\sin\left(\frac{2\pi(t+1)}{c}\right)-\sin\left(\frac{2\pi(t)}{c}\right)\right] .
			\end{align*}
			\end{Proposition}
			\textit{Preuve:} Il suffit d'appliquer \ref{Esp_Accroissements}.\\
			
			Dans la proposition \ref{Intens_Periodique} on interprète $a$ comme étant une tendance cyclique (par exemple annuelle), $b$ comme étant l'intensité des cycles et $c$ comme étant la durée d'un cycle complet. \\
			
			Il s'ensuit que la fonction de vraisemblance est
			\begin{align*}
			\mathcal{L}\left( x,a,b,t \right) 
				&= \prod_{i=1}^{n} \frac{ \left(  a-\frac{bc}{2\pi}\left[\sin\left(\frac{2\pi(t+1)}{c}\right)-\sin\left(\frac{2\pi(t)}{c}\right)\right]  \right)^{x_i} 		e^{-\left(a-\frac{bc}{2\pi}\left[\sin\left(\frac{2\pi(t+1)}{c}\right)-\sin\left(\frac{2\pi(t)}{c}\right)\right] \right)} }{x_i}.\\
			\end{align*}
			
			Encore ici, le plus simple est d'utiliser la méthode algorithmique décrite dans le code informatique \ref{Code_Parametres_PoissSaisonnier}.
			
			\begin{Code} \label{Code_Parametres_PoissSaisonnier}
			\begin{verbatim}	
			
			intensite_cos <- function(t,para){
			para[1] * t - para[2] * sin(2 *pi *t /para[3])
			}
			
			neg_log_vrais <- function(para){
			-sum( log( dpois(data, intensite_cos(t,para) - intensite_cos(t-1,para) )))
			}
			
			mle_nonhomo_cos <- constrOptim(c(20,10,8), neg_log_vrais, grad = NULL, 
			ui = diag(3), ci = c(0,0,0),outer.eps = .Machine$double.eps)
			\end{verbatim}
			\end{Code}