\section{Conclusion}\label{Sect_Conclusion}
To summarize, we were able to develop a functional and practical model for December 2019 and thus delivered 2019 year end predictions to compare with the corporative actuarial department booked numbers. The results were positive and the model is mathematically simple and interpretable. However, we were not fully satisfied with the model and wanted to increase accuracy and consistency. After a few months, we manage to even further reduced the error, although we still have some potential areas of improvement. The imputation method can be potentially improved by taking into account the month of loss, the cause of missing values (accident type, etc.) and other variables. Maybe simply removing the missing values for the factor calculation might be feasible. Systematic overestimation and underestimation depending on the region is the biggest concern and we still have to further investigate the cause. However, it might be related to the observed data trends.
In addition, we might rapidly reach a limit for this simplistic model. We consider developing a more advanced machine learning model for a claim by claim prediction approach.